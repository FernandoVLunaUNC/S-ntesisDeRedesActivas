\section{Introducción}

El presente trabajo se centra en el diseño, análisis y caracterización 
de amplificadores compuestos utilizando tecnologías VFA (Voltage Feedback Amplifier) y CFA 
(Current Feedback Amplifier). Este ejercicio aborda el comportamiento dinámico de sistemas 
amplificadores, con especial énfasis en la compensación de polos y la optimización de la 
respuesta en frecuencia.\\

El estudio incluye configuraciones VFA-VFA y VFA-CFA, evaluando su desempeño frente a criterios 
como planicidad de módulo, margen de fase y ancho de banda. Se busca comprender cómo la ubicación 
de polos y ceros afecta la estabilidad y la fidelidad del sistema, y cómo puede mejorarse mediante 
técnicas de compensación activa.\\

La metodología combina análisis teórico, simulación en PSPICE, implementación práctica en laboratorio 
y comparación de resultados, permitiendo validar el diseño y extraer conclusiones sobre su aplicabilidad 
en sistemas electrónicos de precisión.
