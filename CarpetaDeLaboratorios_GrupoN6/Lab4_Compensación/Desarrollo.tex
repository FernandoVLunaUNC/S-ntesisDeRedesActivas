\section{Desarrollo}

A continuación se presentan los circuitos a analizar en este laboratorio.
Se trata de dos circuitos una de Amplificadores Operacionales en dos etapas, 
y otra de diseño de una Fuente de Tensión de Corriente Continua.

Dentro del desarrollo de los circuitos se harán:
\begin{itemize}
    \item Realizar una breve introducción teórica.
    \item Análisis del circuito.
    \item Realizar el desarrollo numérico y analítico.
    \item Realizar las simulaciones en LTspice.
    \item Realizar el armado y mediciones de laboratorio.
    \item Comprar los resultados obtenidos calculados, simulados y medidos.
\end{itemize}

\subsection{Circuito 1}

Se presenta a continuación el circuito a analizar. La figura 
\autoref{fig:circuito1} muestra el esquema del circuito compuesto 
por dos Amplificadores Operacionales (AO) en cascada que deberá
ser diseñado y analizado para obtener una \textbf{Ganancia Global} de $A_vf=20dB$,
compensándolo para que opere en \textbf{Máxima Planicidad de Módulo} 
($M_\varphi = 65°$ y $Q_p = 0.707$).


\begin{figure}
    \centering
    \begin{circuitikz}
        \draw (0, 0) to [american resistor, l=$R_i$] (2,0);
        \node[ground] at (0,0){};
        \node[op amp, yscale=-1](N1) at (3.2, 0.493){} node[anchor=center] at (N1.text){$LM741$};
        \draw (2,-1) to [american resistor, l=$R_f$] (4,-1) -- (4,0) to [short, -*] (N1.out) -- (2,0) to [short, -*] (N1.-);
        
        

    \end{circuitikz}
    \caption{Esquema del circuito compuesto con dos amplificadores operacionales en cascada}
    \label{fig:circuito1}
\end{figure}

\subsubsection{VFA-VFA}

