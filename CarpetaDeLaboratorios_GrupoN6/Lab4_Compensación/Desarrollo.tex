\section{Desarrollo}

A continuación se presentan los circuitos a analizar en este laboratorio.
Se trata de dos circuitos una de Amplificadores Operacionales en dos etapas, 
y otra de diseño de una Fuente de Tensión de Corriente Continua.

Dentro del desarrollo de los circuitos se harán:
\begin{itemize}
    \item Realizar una breve introducción teórica.
    \item Análisis del circuito.
    \item Realizar el desarrollo numérico y analítico.
    \item Realizar las simulaciones en LTspice.
    \item Realizar el armado y mediciones de laboratorio.
    \item Comprar los resultados obtenidos calculados, simulados y medidos.
\end{itemize}

\subsection{Circuito 1}

Se presenta a continuación el circuito a analizar. La figura 
\autoref{fig:circuito1} muestra el esquema del circuito compuesto 
por dos Amplificadores Operacionales (AO) en cascada que deberá
ser diseñado y analizado para obtener una \textbf{Ganancia Global} de $A_vf=20dB$,
compensándolo para que opere en \textbf{Máxima Planicidad de Módulo} 
($M_\varphi = 65°$ y $Q_p = 0.707$).


\begin{circuitikz}[american]
    \centering
    % Entrada Vin
    \draw (0,0) node[left]{$V_{in}$} to[short, o-] (1,0)
        to[R=$R_i$] (3,0) coordinate (in1);

     % Primer AO (A01)
    \draw (in1) to[short] (3,1)
        node[op amp, anchor=-] (op1) {}
        (op1.out) to[short] ++(1,0) coordinate (out1) node[right]{$V_{mid}$};
    \draw (op1.+) -- ++(0,-0.5) node[ground]{};
    \draw (out1) |- (in1) to[R=$R_f$] (out1);

    % Segundo AO (A02)
    \draw (out1) to[R=$R_1$] ++(0,-2) coordinate (in2);
    \draw (in2) to[short] ++(1.5,0)
        node[op amp, anchor=-] (op2) {}
        (op2.out) to[short] ++(1,0) node[right]{$V_{out}$};
    \draw (op2.+) -- ++(0,-0.5) node[ground]{};
    \draw (op2.-) to[R=$R_2$] ++(0,-1) node[ground]{};
\end{circuitikz}

\subsubsection{VFA-VFA}

